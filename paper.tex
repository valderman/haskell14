%-----------------------------------------------------------------------------
%
%               Template for sigplanconf LaTeX Class
%
% Name:         sigplanconf-template.tex
%
% Purpose:      A template for sigplanconf.cls, which is a LaTeX 2e class
%               file for SIGPLAN conference proceedings.
%
% Guide:        Refer to "Author's Guide to the ACM SIGPLAN Class,"
%               sigplanconf-guide.pdf
%
% Author:       Paul C. Anagnostopoulos
%               Windfall Software
%               978 371-2316
%               paul@windfall.com
%
% Created:      15 February 2005
%
%-----------------------------------------------------------------------------


\documentclass[preprint]{sigplanconf}

% The following \documentclass options may be useful:

% preprint      Remove this option only once the paper is in final form.
% 10pt          To set in 10-point type instead of 9-point.
% 11pt          To set in 11-point type instead of 9-point.
% authoryear    To obtain author/year citation style instead of numeric.

\usepackage{amsmath}


\begin{document}

\special{papersize=8.5in,11in}
\setlength{\pdfpageheight}{\paperheight}
\setlength{\pdfpagewidth}{\paperwidth}

\conferenceinfo{ICFP '14}{Month d--d, 20yy, City, ST, Country} 
\copyrightyear{20yy} 
\copyrightdata{978-1-nnnn-nnnn-n/yy/mm} 
\doi{nnnnnnn.nnnnnnn}

% Uncomment one of the following two, if you are not going for the 
% traditional copyright transfer agreement.

%\exclusivelicense                % ACM gets exclusive license to publish, 
                                  % you retain copyright

%\permissiontopublish             % ACM gets nonexclusive license to publish
                                  % (paid open-access papers, 
                                  % short abstracts)

\titlebanner{banner above paper title}        % These are ignored unless
\preprintfooter{short description of paper}   % 'preprint' option specified.

\title{Type-Safe Client-Server Communication in Web Applications}
\subtitle{Subtitle Text, if any}

\authorinfo{Koen Claessen ans Anton Ekblad}
           {Chalmers University of Technology}
           {\{koen,antonek\}@chalmers.se}

\maketitle

\begin{abstract}
This is the text of the abstract.
\end{abstract}

\category{CR-number}{subcategory}{third-level}

% general terms are not compulsory anymore, 
% you may leave them out
%\terms
%term1, term2
%
\keywords
web

\section{Introduction}

Development of web applications using conventional means is often no task for
the faint of heart. Code written in Javascript, the only widely supported
language for client-side web applications, is often confusing and error-prone,
much due to the language's lack of modularity, encapsulation facilities and
type safety.

Worse, most web applications, being intended to facilitate communication, data
storage and other tasks involving some centralized resource, also require a
significant server component. This component is usually implemented as a
completely separate program, and communicates with the client code over some
network protocol.

Having your application rely on some network protocol for
its internal functionality involves writing a lot of boilerplate network code
and some serious advance planning of your application's features to be able to
design a suitable protocol, hampering incremental development. Were we to
discount this initial design burden, ensuring that the protocol is used
correctly throughout the code base as the application evolves and gains new
features would still put an ever increasing burden on the developer.

This state of things is not a concious design choice - most web applications
are conceptually a single entity, not two programs which just happen to talk
to each other over a network - but a consequence of there being a large,
distributed network between the client and server parts.
However, such implementation details should not be allowed to dictate the way
we structure and reason about our applications - clearly, an abstraction is
called for.

In this paper, we propose a programming model in which web applications are
written as a single entity from which client and server binaries are generated
during compilation. Type annotations in the source program control which parts
are executed on the server and which are executed on the client, and the two
communicate using type safe RPC calls.

Other approaches exist which attempt to tackle the same issue[CITE: STRATIFIEDJS, DUETTO, SUNROOF]
which solve some, but not all, of the associated problems. In particular, our
solution has the following properties:

\begin{itemize}
  \item Type safety: StratifiedJS, being a close relative of the original
        Javascript language, provides type safety for neither RPC calls nor the
        core language and Duetto, being a C++ dialect, is neither memory safe
        nor type safe.
  \item Modularity: our programming model is built as a library on top of
        Haskell and is thus easily portable to any Haskell-to-Javascrit
        compiler. In contrast, StratifiedJS and Duetto are compilers in their
        own right, not usable together with other similar tools.
  \item Purely functional semantics: in contrast to StratifiedJS, Duetto and
        Sunroof, our programming model enables the use of the full Haskell
        language both on the server and the client.
  \item Code reuse: with Sunroof, client code is written in an embedded
        language whereas server code is written in full Haskell, necessitating
        duplicate implementations of functions which may be of use to both the
        client and the server parts. In our solution, once a function is
        written it is usable on both the server and the client.
\end{itemize}


\section{The Haste compiler}
\section{A seamless programming model}
\section{Implementation}
\section{Discussion and related work}

\appendix
\section{Appendix Title}

This is the text of the appendix, if you need one.

\acks

Acknowledgments, if needed.

% We recommend abbrvnat bibliography style.

\bibliographystyle{abbrvnat}

% The bibliography should be embedded for final submission.

\begin{thebibliography}{}
\softraggedright

\bibitem[Smith et~al.(2009)Smith, Jones]{smith02}
P. Q. Smith, and X. Y. Jones. ...reference text...

\end{thebibliography}


\end{document}
